\documentclass[11pt]{article}
\usepackage{fullpage}
\usepackage{amssymb} %previously, latexsym      % not for suns
\usepackage{amsfonts}                           % not for suns 
\usepackage{latexsym}
\usepackage{amsmath}
\usepackage{graphicx} % needed for graphics
\usepackage{multicol} % allows dynamic column specification in document
\usepackage{multirow}
\usepackage{listings}
\usepackage{mdwlist}
\usepackage{longtable}
\usepackage{verbatim} %lets me comment out multiple lines with \begin{comment}
\usepackage{natbib}
\usepackage{indentfirst}
\usepackage{booktabs} % gives better spacing in tables
\usepackage{bm}     % provides bold greek letters in math mode
\usepackage{float}  % if using H in table figures forces location of float
\usepackage{wrapfig} % lets me put figures in margin
\usepackage{setspace} % this gives me the ability to double space
\usepackage{array} % this allows me to define certain characteristics for a single column within a table
\usepackage{tabularx}% this allows me to redefine text alignment within in a column.
\usepackage{rotating} %This allows me to rotate floats
\usepackage{appendix} % this gives an appendix title to the section
\usepackage{blindtext}
\usepackage{amsthm}
\usepackage{color}
\usepackage{enumitem}
\usepackage{wrapfig}
\usepackage[font=small,skip=-10pt]{caption}
\usepackage{ulem}
\usepackage[left=1in,top=1in,right=1in,bottom=1.25in,nohead]{geometry}


\captionsetup{belowskip=-25pt}

%\setlength{\parindent}{0in}

\bibliographystyle{asa}
\bibpunct{(}{)}{;}{a}{}{,}

%\newcommand{\qed}{\nobreak \ifvmode \relax \else
%      \ifdim\lastskip<1.5em \hskip-\lastskip
%      \hskip1.5em plus0em minus0.5em \fi \nobreak
%      \vrule height0.75em width0.5em depth0.25em\fi}



%\input Macros




\begin{document}



\noindent  {\bf \Large Facilities, Equipment, and Other Resources} 

\vspace{5pt}

%\section{Brigham Young University}

\noindent Dr.\ David B.\ Dahl is a member of the Department of Statistics at Brigham Young University (BYU).  The department currently occupies a wing of the West View Building located in the heart of the Provo Utah campus.  Graduate students are provided shared office space there in.  The department also has state of the art conference rooms that contain teleconferencing equipment, projectors, etc.  In addition to the described physical facilities, the computing facilities of the Department of Statistics are state of the art.  The following is a list that details the computing environment in the BYU Department of Statistics:
\begin{enumerate}
\itemsep-0.1em 

\item A dedicated server room
\begin{enumerate}
\itemsep-0.1em 
\item Six Racks
\item Redundant UPS
\item Core network switch with GE and 10G connectivity
\end{enumerate}

\item Blade Server
\begin{enumerate}
\itemsep-0.1em 
\item 6 slots for blades
\item 2 blades each with 32 CUPs and 64 GB RAM
\item SAN attached storage, 6.5 TB
\end{enumerate}

\item Network attached storage
\begin{enumerate}
\itemsep-0.1em 
\item Highly redundant 3 node cluster
\item 32 TB disk
\end{enumerate}

\item 18 research servers
\begin{enumerate}
\itemsep-0.1em 
\item Each server has between 128G to 786G of RAM
\item Total of 1,392 CPU threads
\item Total of 12TB internal disk storage
\end{enumerate}

\item A hyper-converged infrastructure supporting file and web services
\begin{enumerate}
\itemsep-0.1em 
\item Highly redundant 2 node cluster
\item 8 core, 128 GB RAM, 14 TB disk
\item Remote DR node
\end{enumerate}

\item Computer support staff
\begin{enumerate}
\itemsep-0.1em 
\item Full time Systems Adminstrator
\item Two part computer support specialists
\end{enumerate}

\end{enumerate} 

\noindent Additionally there will be access to the Fulton Super Computing Lab
\begin{enumerate}
\itemsep-0.1em 
\item 21,552 CPU cores, 78 TB of memory, 972 compute nodes
\item 2 petabyes of high performance storage
\item Red Hat Enterprise linux
\end{enumerate}

\end{document}



